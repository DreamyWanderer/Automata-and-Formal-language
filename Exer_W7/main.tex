\documentclass[12pt, a4paper]{article}
\usepackage{fancyhdr, url}
\usepackage{ragged2e}
\usepackage[pdftex]{graphicx}
\usepackage[margin=1in]{geometry}
\usepackage[vietnamese]{babel}

\usepackage{pgffor}
\usepackage{ifthen}
\usepackage{listofitems}

\usepackage{tikz}
\usetikzlibrary {automata, positioning, arrows}

\usepackage{amsmath, amssymb}
\usepackage{cases}
\usepackage{bm}

\newcommand{\fDFA}[3]{
	\delta_D(\{
	\foreach[count=\i] \j in {#1} {
		\ifthenelse{\i < #2}{s_\j, }{s_\j} 
	}
	\}, #3)
}

\newcommand{\GDerivative}[2] {
	\foreach[count=\i] \j in {#1} {
		\j
		\ifthenelse{\i < #2}{ \Rightarrow }{}
	}
}


\begin{document}
	
	\pagenumbering{gobble}

	\begin{titlepage}
		\centering
		\large
		ĐẠI HỌC KHOA HỌC TỰ NHIÊN, ĐẠI HỌC QUỐC GIA TP.HCM\\[.1in]
		KHOA CÔNG NGHỆ THÔNG TIN\\BỘ MÔN KHOA HỌC MÁY TÍNH\\
		\vfill
		\huge AUTOMATA VÀ\\NGÔN NGỮ HÌNH THỨC\\[.1in]
		\LARGE BÀI TẬP CHƯƠNG 6\\
		\vfill
		\RaggedRight
		\large
		Sinh viên thực hiện: Nguyễn Thế Hoàng (MSSV: 2012 0090)\\[.1in]
		Giáo viên phụ trách: Nguyễn Thanh Phương - Lê Ngọc Thành\\[.2in]
		\Centering
		BÀI TẬP MÔN HỌC - AUTOMATA VÀ NGÔN NGỮ HÌNH THỨC\\[.1in]
		HỌC KỲ II - NĂM HỌC 2022 - 2023
	\end{titlepage}
	
	\newpage	
	
	\pagenumbering{arabic}
	
	\begin{itemize}
		
		\item[ \textbf{Bài 1}]
		
		\begin{itemize}
		
			\item[a.] $S \rightarrow SS \mid aS \mid b$
			
			Một dẫn xuất khả dĩ đáp ứng yêu cầu là:
			
			$ \GDerivative{S, aS, aaS, aaaS, aaaSS, aaaSSS, aaaSSSS, aaaSSSSS, aaaSSSSSS, aaaSSSSSSS, aaabbbbbbb}{11} $			
			
			$\mathcal{L}(\mathcal{G})$ bao gồm các chuỗi chứa tùy ý số lượng kí tự a và b, nhưng không có kí tự a nào đứng sau b, và không được là $\varepsilon$.
			
			\item[b.] $S \rightarrow aSaa \mid B, B \rightarrow bbBdd \mid C, C \rightarrow bd$
			
			Một dẫn xuất khả dĩ đáp ứng yêu cầu là:
			
			$\GDerivative{aSaa, aaSaaaa, a^2Ba^4, a^2bbBdda^4, a^2bbbbBdddda^4, a^2b^4Cd^4a^4, a^2b^5d^5a^4}{7}$
			
			$\mathcal{L}(\mathcal{G}) = \{ a^ib^jd^ja^{2i}: i, j \in \mathbb{N}^* \}$
			
			\item[f.] $S \rightarrow bSaS \mid aSbS \mid \varepsilon$
			
			Một dẫn xuất khả dĩ đáp ứng yêu cầu là:
			
			$\GDerivative{bSaS, bbSaSaS, bbaSbSaSaS, bbabSaSbSaSaS, bbabaSbSaSbSaSaS, bbabababaa}{6}$
			
			$\mathcal{L}(\mathcal{G})$ bao gồm các chuỗi gồm: tùy ý số lượng kí tự a, b xen kẽ nhau sao cho $|a| = |b|$; hoặc chuỗi $\varepsilon$.
			
		\end{itemize}
		
		\item[ \textbf{Bài 2} ]
		
		\begin{itemize}
			
			\item[a.] $\mathcal{L} = \{ a^nx: n \in \mathbb{N} \wedge x \in \{a, b\}^* \wedge |x| = n \}$
			
			Tập luật sinh $P$ của văn phạm $\mathcal{G}_{\mathcal{L}}$ là: 
			
			$\{
			S \rightarrow aSX \mid \varepsilon,
			X \rightarrow a \mid b
			\}$			
			
			\item[b.] $\mathcal{L} = \{ a^nb^na^mb^m: m, n \in \mathbb{N} \}$
			
			Tập luật sinh $P$ của văn phạm $\mathcal{G}_{\mathcal{L}}$ là: 
			
			$\{
			A \rightarrow aSbX \mid \varepsilon,
			S \rightarrow aSb \mid \varepsilon,
			X \rightarrow aSb \mid \varepsilon
			\}$	
			
			\item[c.] $\mathcal{L} = \{ a^nb^{n + m}b^m: m, n \in \mathbb{N} \}$
			
			Tập luật sinh $P$ của văn phạm $\mathcal{G}_{\mathcal{L}}$ là: 
			
			$\{
			A \rightarrow aSbX \mid \varepsilon,
			S \rightarrow aSb \mid \varepsilon,
			X \rightarrow bbX \mid \varepsilon
			\}$
			
			\item[g.] $\mathcal{L} = \{ a^nb^mb^{2n + m}: m, n \in \mathbb{Z}^+ \}$
			
			Tập luật sinh $P$ của văn phạm $\mathcal{G}_{\mathcal{L}}$ là: 
			
			$\{
			A \rightarrow aSbbX,
			S \rightarrow aSbb \mid \varepsilon,
			X \rightarrow bb \mid \varepsilon
			\}$
			
			\item[h.] $\mathcal{L} = \{ w \in \{a, b\}^*: |w|_a = 2|w|_b \}$
			
			Tập luật sinh $P$ của văn phạm $\mathcal{G}_{\mathcal{L}}$ là: 
			
			$\{
			S \rightarrow SbSaSaS \mid SaSbSaS \mid SaSaSbS \mid \varepsilon
			\}$
			
		\end{itemize}
			
	\end{itemize}
	
\end{document}
