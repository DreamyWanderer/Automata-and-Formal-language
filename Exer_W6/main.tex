\documentclass[12pt, a4paper]{article}
\usepackage{fancyhdr, url}
\usepackage{ragged2e}
\usepackage[pdftex]{graphicx}
\usepackage[margin=1in]{geometry}
\usepackage[vietnamese]{babel}

\usepackage{pgffor}
\usepackage{ifthen}
\usepackage{listofitems}

\usepackage{tikz}
\usetikzlibrary {automata, positioning, arrows}

\usepackage{amsmath, amssymb}
\usepackage{cases}
\usepackage{bm}

\newcommand{\fDFA}[3]{
	\delta_D(\{
	\foreach[count=\i] \j in {#1} {
		\ifthenelse{\i < #2}{s_\j, }{s_\j} 
	}
	\}, #3)
}

\newcommand{\fNFA}[2]{
	\delta_N(s_{#1}, #2)
}

\newcommand{\convNFAtoDFA}[5]{
	$\fDFA{#1}{#2}{#3} = 
	\foreach[count=\i] \j in {#1} {
		\fNFA{\j}{#3}
		\ifthenelse{\i < #2}{\cup}{}
	} = \{
	\foreach[count=\i] \j in {#4} {
		\ifthenelse{\i < #5}{s_\j, }{s_\j}
	}\}$
}

\newcommand{\epsCLOSE}[1]{
	\varepsilon\text{CLOSE}\left(\ #1 \ \right)
}

\newcommand{\convEpsNFAtoDFA}[7]{
	$\fDFA{#1}{#2}{#3} = \\
	\epsCLOSE{
	\foreach[count=\i] \j in {#1} {
		\fNFA{\j}{#3}
		\ifthenelse{\i < #2}{\cup}{}
	}} =
	\epsCLOSE{
	\{
	\foreach[count=\i] \j in {#4} {
		\ifthenelse{\i < #5}{s_\j, }{s_\j}
	} \}
	} = 
	\{
	\foreach[count=\i] \j in {#6} {
		\ifthenelse{\i < #7}{s_\j, }{s_\j}
	}
	\}$
}

\begin{document}
	
	\pagenumbering{gobble}

	\begin{titlepage}
		\centering
		\large
		ĐẠI HỌC KHOA HỌC TỰ NHIÊN, ĐẠI HỌC QUỐC GIA TP.HCM\\[.1in]
		KHOA CÔNG NGHỆ THÔNG TIN\\BỘ MÔN KHOA HỌC MÁY TÍNH\\
		\vfill
		\huge AUTOMATA VÀ\\NGÔN NGỮ HÌNH THỨC\\[.1in]
		\LARGE BÀI TẬP CHƯƠNG 3 - PHẦN 2\\
		\vfill
		\RaggedRight
		\large
		Sinh viên thực hiện: Nguyễn Thế Hoàng (MSSV: 2012 0090)\\[.1in]
		Giáo viên phụ trách: Nguyễn Thanh Phương - Lê Ngọc Thành\\[.2in]
		\Centering
		BÀI TẬP MÔN HỌC - AUTOMATA VÀ NGÔN NGỮ HÌNH THỨC\\[.1in]
		HỌC KỲ II - NĂM HỌC 2022 - 2023
	\end{titlepage}
	
	\newpage	
	
	\pagenumbering{arabic}
	
	\begin{itemize}
		
		\item[ \textbf{Bài 5} ]
		
		\item[ \textbf{Bài 6} ]
		
		\begin{itemize}
		
			\item[a.] Hệ phương trình là:
		
				\begin{numcases}{}
				\bm{ x_0 = bx_0 + ax_1 }\\
				\bm{ x_1 = (a + b)x_2 }\\
				\bm{ x_2 = \varepsilon + (a + b)x_1 }
				\end{numcases}		
				
				Áp dụng định lý, và thế giá trị $x_1$ vào (3), ta có:
				
				\begin{align*}
				\bm{ x_2 = \varepsilon + (a + b)(a + b)x_2 = ((a + b)(a + b))^* }
				\end{align*}
				
				Thế giá trị $x_2$ vào (2):
				
				\begin{align*}
				\bm{ x_1 = (a + b)((a + b)(a + b))^* }
				\end{align*}
				
				Thế giá trị $x_1, x_2$ vào (1):
				
				\begin{align*}
				\bm{ x_0 = bx_0 + a(a + b)((a + b)(a + b))^*  = b^*a(a + b)((a + b)(a + b))^*}
				\end{align*}
				
				\bm{$x_0$} chính là biểu thức chính quy cần tìm vì $q_0$ là trạng thái bắt đầu.
				
			\item[b.] Hệ phương trình là:
			
				\begin{numcases}{}
				\bm{ x_0 = ax_0 + ax_1 }\\
				\bm{ x_1 = bx_0 + bx_1 + ax_2 }\\
				\bm{ x_2 = \varepsilon + ax_1 }
				\end{numcases}
				
				Thế giá trị $x_2$ vào (5):
				
				\begin{align*}
				\bm{ x_1 = bx_0 + bx_1 + aax_1}
				\end{align*}
		
		\end{itemize}
	
	\end{itemize}
	
\end{document}
